% This is a LaTeX template kindly taken from Jernej Debevec.
% Provided by Miha Muskinja for the purpose of the seminar I in the 1st year
% of the 2nd cycle of the study of physics at the Faculty of Mathematics and Physics, University of Ljubljana.

% Set the document class and options
\documentclass[10pt, titlepage, a4paper]{article}
\usepackage[a4paper, inner=2.5cm, outer=2.5cm, top=2.25cm, bottom=2.25cm]{geometry}
\usepackage{graphicx}
\usepackage{hyperref}
\usepackage{wrapfig}
\hypersetup{colorlinks=true}

% Load the natbib package for citation style
\usepackage{natbib}


\newcommand{\bb}[1]{\mathbf{#1}}

% Start the document
\begin{document}

% The title page
\begin{titlepage}
{\centering
\includegraphics[width=6cm]{logo_fmf.pdf}

\vspace{0.8cm}
{\small Department of Physics}

\vspace{5cm}
\vspace{0.5cm}
{\huge\textbf{Metropolis-Hastings Algorithm}} \\
\vspace{0.5cm}
{\large\textbf{8. Task for Model Analysis I, 2023/24}}

\vfill
\textbf{Author:} Marko Urbanč \\
\textbf{Professor:} Prof. Dr. Simon Širca \\ 
\textbf{Advisor:} doc. dr. Miha Mihovilovič \\

\vspace{1cm}
Ljubljana, July 2024 \\
}
\vspace{3cm}
\end{titlepage}

% Add table of conents
\hypersetup{pageanchor=true}
\pagenumbering{roman}
\setcounter{page}{2}
\tableofcontents
\vspace{1cm}

% Proceed with the main body
\pagenumbering{arabic}

\section{Introduction}
We're continuing our exploration into random numbers and their applications. Previously we had a look at Monte Carlo 
sampling. Today we're going to delve into the Metropolis-Hastings algorithm, which can be thought of as Monte Carlo 
sampling with a few extra steps. Since our end goal is to simulate the relaxation of a lattice of spins in a magnetic
field, we'll take the Ising model as our physical context. \\

We know from statistical physics that the 2D Ising model relaxes to a state of minimum energy. We can simulate this 
relaxation by flipping spins at random and accepting or rejecting the new state based on the change in energy. We can have 
a negative change of energy which we can call a \textit{good move} or a positive change of energy which we can call a
\textit{bad move}. The added twist with this algorithm is that while we always accept the new state if we make a good move, we 
also sometimes accept a new state after a bad move. This is the key to the Metropolis-Hastings algorithm. Given our system the probability of
accepting a bad move is given by the Boltzmann factor and the temperature of the system:
%
\begin{equation}
    \label{eq:bad-accept-prob}
    P_{\mathrm{bad\>accept}} = \exp\left(-\frac{\Delta E}{kT}\right)\>,
\end{equation}
%
where $\Delta E$ is the change in energy, $k$ is the Boltzmann constant and $T$ is the temperature. Why exactly this works 
is a bit more involved and probably out of the scope of this report however a dedicated reader can find more information in this 
well written blog post by Gregory Gundersen of Princeton University \cite{Gundersen_2019}. \\



\section{Task}
\section{Solution Overview}
\section{Results}
\section{Conclusion and Comments}

% Add references
\newpage
\bibliographystyle{unsrt}
\bibliography{mod108}

% End document
\end{document}
